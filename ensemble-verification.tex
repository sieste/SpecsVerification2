\documentclass[article]{jss}
\usepackage[utf8]{inputenc}
\usepackage{amsmath}
\usepackage{dsfont}
\usepackage{tabularx}

%%%%%%%%%%%%%%%%%%%%%%%%%%%%%%
%% declarations for jss.cls %%%%%%%%%%%%%%%%%%%%%%%%%%%%%%%%%%%%%%%%%%
%%%%%%%%%%%%%%%%%%%%%%%%%%%%%%

%% almost as usual
\author{Stefan Siegert\\University of Exeter}
\title{Verification of Ensemble Forecasts: Ensemble-adjusted Scores, Comparative Verification, and Uncertainty Quantification Implemented in the \proglang{R} Package \pkg{SpecsVerification}}
\Shorttitle{\pkg{SpecsVerification}: R Functions for Ensemble Verification} %% a short title (if necessary)
\Plainauthor{Stefan Siegert} %% comma-separated
\Plaintitle{Forecast Verification Methods Implemented in the R package SpecsVerification: Ensemble Scores, Comparative Verification, and Uncertainty Quantification}

%% an abstract and keywords
\Abstract{

Forecast verification, the comparison of retrospective forecasts to observations, is a common task at institutions that develop and issue forecasts, such as climate centers.
To assess forecast uncertainty , ensembles of forecasts initialised from perturbed initial conditions are routinely issued.
In recent years, advances have been made in statistical methodology for ensemble forecast verification, in particular to estimate the finite-ensemble effect on verification scores.
This paper summarises statistical methodology to account for finite-ensemble effects in ensemble verification, to compare the quality of ensemble forecasts with different numbers of ensemble members, and to quantify uncertainty in forecast verification results.
Implementations of the methods are freely available in the \proglang{R} package \pkg{SpecsVerification}.

}
\Keywords{ensemble forecasting, forecast verification, finite-ensemble effect, comparative verification, uncertainty quantification, \proglang{R}}
\Plainkeywords{keywords, comma-separated, not capitalized, R} %% without formatting
%% at least one keyword must be supplied

%% publication information
%% NOTE: Typically, this can be left commented and will be filled out by the technical editor
%% \Volume{50}
%% \Issue{9}
%% \Month{June}
%% \Year{2012}
%% \Submitdate{2012-06-04}
%% \Acceptdate{2012-06-04}

%% The address of (at least) one author should be given
%% in the following format:
\Address{
  Stefan Siegert\\
  Exeter Climate Systems\\
  College for Engineering, Mathematics, and Physical Sciences\\
  University of Exeter\\
  Exeter, EX4 4QF, United Kingdom\\
  E-mail: \email{Stefan.Siegert@exeter.ac.uk}\\
  URL: \url{http://emps.exeter.ac.uk/mathematics/staff/ss610}
}


%% end of declarations %%%%%%%%%%%%%%%%%%%%%%%%%%%%%%%%%%%%%%%%%%%%%%%

% initialise R session and knitr
